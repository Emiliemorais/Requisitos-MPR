\section[Escolha da Ferramenta]{Escolha da Ferramenta}
A ferramenta escolhida foi o TraceCloud. Como já dito, o TraceCloud é uma ferramenta web, portanto o grupo se compromete e assume os riscos de conseguir apresentar as documentações geradas por ela independente da qualidade da internet em dias de apresentação. A vantagem é que todos componentes do grupo ganham uma maior autonomia com todos podendo alterar os documentos sempre que necessário, pois em casos de ferramentas offline, certamente seria necessário a delegação da função para um único aluno ficar responsável por gerenciar tais requisitos, o que é um grande risco, visto que o arquivo do computador pode sofrer um dano assim como o próprio sistema local.

Como explicado e avaliado na tabela acima, as ferramentas avaliadas pelo grupo receberam notas de 0 a 5 em algumas características previamente estudadas, e comparando com as outras ferramentas avaliadas, em vários pontos ela foi melhor pontuada. Em especial, no tópico que diz respeito ao processo de Engenharia de Requisitos da Empresa Junior de Engenharia de Energia, a ferramenta possui uma opção de escolha entre metodologias ágeis e tradicionais, sendo que cada uma recebe um melhor tratamento e as instruções que melhor se adequam ao trabalho. Além de fornecer um suporte ao SCRUM, tal ferramenta pode documentar as datas das modificações que os requisitos recebem ao longo do projeto garantindo total rastreabilidade dos mesmos.

Por fim, outro ponto que deseja-se utiizar bastante, é o já dito anteriormente, delegamento de funções por sprint. Tendo em vista que a cada período de tempo algumas tarefas deverão ser feitas, hora individuais e hora em grupo, para obtenção de um maior controle, organização e para reforço do cronograma (usado com outra ferramenta), projeta-se o uso de tal funcionalidade.
