\chapter{Questões e respostas das entrevistas}
\label{questionario}

\section{Primeira entrevista}
Na primeira entrevista foram coletadas respostas de dois \textit{stakeholders}.

\begin{enumerate}
	\item \textbf{O sistema é só para a Matriz, ou também é voltado para o cliente?}
		\subitem Só para a Matriz.

	\item \textbf{Quais são os recursos tecnológicos que a matriz possui? (Computadores, notebooks, tablets, celular)}
		\subitem A empresa não possui espaço físico, os recursos são pessoais. A empresa não fornece equipamentos.
		\subitem Os colaboradores da empresa possuem computadores e tablets que são de uso pessoais e os utilizam também para fins empresariais.
	\item \textbf{Como eles compartilham as informações acerca dos projetos e clientes?}
	      \subitem Armazenamento na nuvem.
	\item \textbf{Quem são os responsáveis pela divulgação dos serviços (papéis dentro da empresa)?}
		\subitem Diretor de Marketing e dois consultores que podem ser de Marketing.
		\subitem Equipe de marketing, departamento de marketing (Diretor de marketing, apenas ele na equipe)
	\item \textbf{Como é realizado o processo de escolha dos clientes a serem contatados através de ligações?}
		\subitem Olhando em catálogos e identificando a necessidade de atuação previamente definida. É realizada uma análise de perfil.
			
		\subitem Os diretores de reúnem para fazer um planejamento entre os diretores da empresa. Neste planejamento é definido qual o perfil dos clientes a serem contatados.

	\item \textbf{Como é feita a análise de perfil?}
		\subitem Eles tentam identificar problemas dessa empresa, buscam pelo nome, e estudam o tamanho da empresa, tipo de atividade que ela faz, meios de contato com a empresa. 
	Para redução de energia, estudam os gastos da empresa.
	A análise é determinada pelo serviço.
	(Máximo de informação possível).

		\subitem Analisam de acordo com as necessidades do cliente.
	\item \textbf{Como é definido atualmente os meios de divulgação?}
		\subitem Primeiramente, a ideia era disparar nas redes sociais, porém isso não estava dando retorno. A partir disso, eles começaram a fazer ligações para possíveis clientes e fecharam o escopo para gelo e frigorífero. 

		\subitem Possuem um site. Fazem divulgação por redes sociais e por meio de cartazes espalhados dentro do campus da FGA. Por meio de ligações.
		
	\item \textbf{As informações acerca dos clientes contatados são registradas? Se sim, de que forma?}
		\subitem São armazenadas, mas não tem um padrão para isso. É necessário armazenar essas informações.
		\subitem Sim. Em um agenda, um caderno. Não tem um padrão específico de armazenamento.
	\item \textbf{Quem são os responsáveis pelo pós-marketing?}
		\subitem Diretor de Projeto e gerente de projeto e dois consultores de projeto(que estão entrando agora).
		\subitem Gerente de projeto.
	\item \textbf{Como os detalhes do processo são acompanhados hoje, e em que ponto pode melhorar? Pra que?}
		\subitem Eles não acompanham esse processo, porque não tem clientes.
		\subitem Não tem um acompanhamento específico. Não dá pra definir ainda, pois não adquiriram nenhum cliente.
	\item \textbf{Na visita técnica as informações obtidas são registradas? Se sim, como?}
		\subitem Sim, são armazenadas em planilhas ou papéis.
		\subitem Sim. Em planilhas, papéis.
	\item \textbf{A partir do momento que o cliente é contatado, como é acompanhada a sua permanência até o fechamento do contrato?}
		\subitem Nunca executaram o processo.
		\subitem Não existe um responsável de acompanhamento geral. O responsável do processo atual é que está responsabilizado por ele.
	\item \textbf{É interessante manter status do processo?}
		\subitem Sim.
		\subitem Sim.
\end{enumerate}

\section{Segunda entrevista}

\begin{enumerate}
 \item \textbf{Além do sistema que faz o registro dos dados (banco de dados), o sistema se comunicará com outros sistemas?}
 	\subitem Atualmente não. No entanto o sistema deve permitir que haja margem para o recebimento de arquivos gerados.
 \item \textbf{A empresa, atualmente, necessita de que o sistema siga padrões de acessibilidade para deficientes }físicos?
 	\subitem Sim. Pois a empresa possui uma rotatividade alta de funcionários.
 \item \textbf{O sistema será integrado a outro(s) sistema(s)?}
 	\subitem Não. 
 \item \textbf{O sistema precisa ser executado em diferentes plataformas? Sejam elas:  \textit{tablets} ou \textit{smartphones} Android, iOs e Windows; ou computadores com sistema operacional Windows, Mac ou Linux, etc?}
 	\subitem O ideal que o sistema seja acessível por diferentes plataformas, uma solução pra isso seria o sistema ser \textit{Web}.
 \item \textbf{Quais são os navegadores em que o sistema deve funcionar?}
 	\subitem A maioria dos usuários provavelmente usam o Google Chromme. Versões mais recentes do Google Chromme e do Mozilla Firefox.
 \item \textbf{Com relação ao tempo de execução do programa, há alguma necessidade específica em termos de tempo de resposta do sistema?}
 	\subitem Espera-se que não leve mais do que 3 segundos para uma requisição.
  \item \textbf{No que se refere a confiabilidade, que é a capacidade de o sistema lidar com eventos inesperados (a probabilidade de operação livre de falhas de um  programa de computador num ambiente específico durante determinado tempo), quais as necessidades do sistema quanto a isso?}
 	\subitem O ideal é que qualquer exceção sejam tratadas para evitar que o usuário perceba algum erro no sistema.
	Ideal que as informações armazenam os dados preenchidos em alguma memória cache para caso ocorra algum erro no sistema, os dados digitados não sejam perdidos, evitando que o usuário tenha o retrabalho de redigitar.
	Evitar perda de dados importantes do cliente.
 \item \textbf{Qual o maior tempo que o sistema pode permanecer fora do ar após uma falha?}
 	\subitem No máximo um dia.
 \item \textbf{Qual deve ser a disponibilidade do sistema? O sistema deve estar disponível 24 horas por dia, 7 dias por semana?}
 	\subitem O sistema deve estar sempre \textit{online}, mas caso não seja possível, ele deve estar sempre acessível de 6:00 - 00:00.
 \item \textbf{Em relação a usabilidade (capacidade do produto de software ser compreendido, seu funcionamento aprendido, ser operado e ser atraente ao usuário) quais as características o sistema deve ter?}
 	\subitem Fácil utilização, auto-explicativo, campo de descrição de campos dos vários formulários.Utilizar de forma fluida, intuitivo.
 \item \textbf{É necessário um manual do usuário, ou uma ajuda \textit{online}, visto que os funcionários são rotativos? O usuário passará por um treinamento de quanto tempo para utilizar o sistema?}
    \subitem Devido a alta rotatividade dos funcionários, é imprescindível a presença de um manual \textit{online} de auto-ajuda. Duas ou três semanas.
 \item \textbf{No dia a dia de sua operação, o software necessita atender alguma política ou procedimento da empresa? Algum padrão específico? Alguma infraestrutura física que deve ser suportada pelo software?}
    \subitem Não.
 \item \textbf{Há fatores externos ao sistema e ao processo de desenvolvimento que devem ser usados levados em conta tais 
 	 como:
  \subitem a. alguma legislação específica?
  \subitem b. localização geográfica?
  \subitem c. outros?}
 \subitem Não.
 \item \textbf{Como o sistema deve se comportar em relação ao controle de acesso?}
 	\subitem Deve haver um controle de acesso e o usuário não pode se logar em duas máquinas ao mesmo tempo.
\end{enumerate}

