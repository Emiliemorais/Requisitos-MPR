\section[Ferramentas encontradas]{Ferramentas encontradas}
Ao longo dos últimos anos, tem havido uma mudança dramática no gerenciamento de requisitos, nas ferramentas disponíveis no mercado, e seus recursos disponíveis (\citeauthor{beatty}, \citeyear{beatty}).

Foram encontradas diversas ferramentas tais como: Jeremia, Open Source Requirements Management Tool (OSRMT), Xuse, Tiger Pro, TraceCloud e Caliber. Considerando o critério de melhor atender as necessidades do processo de engenharia de requisitos atual, foram selecionadas, entre essas, apenas três para a avaliação. Essas ferramentas estão detalhadas a seguir.


\subsection{Tiger Pro}
O Tiger Pro (Tool to InGest and Elucidate Requirements PROfessional) é uma ferramenta de gerenciamento de requisitos licenciada para o uso educacional que permite que os requisitos, além de também poderem ser adicionados diretamente no programa, possam ser adicionados através de documentos.

“Produz um sumário de informações na elucidação de requisitos, alocação de prioridades, definição de riscos e estimativa de custo em formato de texto ou gráfico.”[\citeauthor{ananias}, \citeyear{ananias}].

Apesar do software ser licenciado para o uso educacional, é possível utilizá-lo para fins comerciais, desde que haja uma autorização do responsável pelo software.

\subsection{TraceCloud}
TraceCloud é um software para gerência de requisitos independente se o projeto possui abordagem adaptativa ou se possui abordagem tradicional. Esse software gerencia as mudanças dos requisitos, desde mudanças de baixo nível até as de mais alto nível. Além disso, o TraceCloud dá um suporte completo de rastreabilidade dos requisitos.

O TraceCloud é bastante flexível ou seja, ele se adapta a qualquer processo de negócio. Vale ressaltar que dentro do contexto de nosso projeto, onde os atributos de requisitos foram outrora definidos, o TraceCloud fornece suporte a cada atributo.

O TraceCloud não é gratuito, mas possui versão gratuita para testes.

\subsection{Caliber}
O Caliber possui uma solução completa para gerenciamento de requisitos que garante a conformidade e alinhamento de desenvolvimento para atender as necessidades de negócios. Ele facilita a colaboração dos Stakeholders, possui uma visualização rica, gerenciamento robusto e uma rastreabilidade dos requisitos bem definida e completa.

O Caliber garante a entrega de um software com maior precisão e que atenda as necessidades dos clientes.

O software permite a participação do usuário de forma direta com o processo aumentando a clareza e precisão. Com isso, o usuário dá feedbacks para garantir que os requisitos não sejam definidos de forma inequívoca.

O Caliber fornece uma rastreabilidade completa dos requisitos para modelar as relações internas e externas dos mesmos.  Essa rastreabilidade suporta as mudanças dos requisitos  e responde como essas mudanças causam impacto no projeto.

O Caliber não é gratuito, mas possui versão de 30 dias para testes.
