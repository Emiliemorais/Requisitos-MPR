\section{Artefatos}

Os artefatos foram definidos de acordo com a proposta do SAFe feita por \citeonline{safe} e do Scrum feita 
por \citeonline{scrum}.

\begin{itemize}
  \item \textit{Backlog} do Programa
    \subitem O \textit{Backlog} do Programa é responsável por manter os épicos, as \textit{features} e os requisitos
    não funcionais. De acordo com \citeonline{safe}, o \textit{Backlog} do Programa mantém apenas as \textit{features} e os requisitos não funcionais,
    todavia para o processo definido a constituição deste \textit{Backlog} foi adaptado e ele passou a manter também
    os épicos. Essa escolha foi realizada para diminuir a quantidade de artefatos gerados. 
    
    Esse artefato é criado na atividade Levantar Épicos e 
    pode ser atualizado nas atividades: Fazer reunião de validação dos épicos, Levantar \textit{Features}, Especificar \textit{Features}, 
Fazer reunião de validação das \textit{Features}, Construir Visão, Construir \textit{Roadmap}, Identificar Requisitos Não Funcionais e durante as atividades
do subprocesso Executar \textit{Release}.
  
  \item Visão
   \subitem O Visão consiste em um documento que contém os requisitos funcionais, requisitos não funcionais, incluindo elementos regulatórios 
   ou outros padrões de conformidade, e qualquer restrição de design. Também contém um panorama da solução a ser desenvolvida, 
   refletindo as necessidades das partes interessadas e os recursos propostos para atender essas necessidades. \cite{safe}.
   No processo definido nesse trabalho é gerado na atividade Construir Visão.
  
  \item \textit{Roadmap}
     \subitem O \textit{Roadmap} consiste na alocação das \textit{features} em \textit{Releases}, através da determinação de datas e priorizações. \cite{safe}.
     O \textit{Roadmap} é definido na atividade: Construir \textit{Roadmap}.

    
  \item \textit{Backlog} do Time
      \subitem O \textit{Backlog} do Time é responsável por manter as histórias levantadas a partir das \textit{features}.  \cite{safe}. No processo definido
      é criado na atividade Escrever histórias na primeira \textit{Release} e pode ser atualizado nas atividades: Escrever histórias, Especificar histórias, Realizar reunião de planejamento 
      da Iteração, Fazer reunião de revisão e retrospectiva da Iteração.
  
  \item \textit{Backlog} da Iteração
        \subitem O \textit{Backlog} da Iteração é responsável por manter as histórias da iteração atual. \cite{safe}.
        No processo definido é criado na atividade Realizar reunião de planejamento 
      da Iteração e pode ser atualizado na atividade de Especificar histórias.
  
  \item Incremento de Software
      \subitem O Incremento de Software consiste na solução gerada a cada Iteração. \cite{scrum}. No processo definido é gerada na atividade de Desenvolver Histórias e validada na 
      atividade de Reunião de Revisão e retrospectiva da Iteração.
\end{itemize}
