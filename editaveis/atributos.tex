\section[Atributos de Requisitos]{Atributos de Requisitos}

De acordo com o \citeonline{openUp}, atributos dos Requisitos são as propriedades de um requisito. Assim como uma entidade qualquer no contexto de desenvolvimento de software possui seus atributos, um requisito também possui os seus. 

De maneira semelhante a uma entidade do tipo pessoa que possui atributos, tais como idade, cor do cabelo e sexo, cada requisito possui uma origem, uma importância relativa e a data em que foi criado, por exemplo. \cite{openUpBasic}.

De acordo com o \citeonline{openUpBasic}, os atributos são uma fonte muito importante de informações sobre os requisitos e têm a intenção de capturar informações adicionais de cada requisito. Além de apenas definir as propriedades de um requisito, se bem definidos, os atributos podem fornecer significantes informações sobre o estado do desenvolvimento de um sistema. Estabelecendo um paralelo com uma consulta em um banco de dados em que se deseja encontrar todos os homens com cabelo castanho e idade maior que 30 anos, é possível consultar os 
atributos de um requisito para encontrar os requisitos de alta prioridade para o cliente nos últimos 30 dias. 
\cite{openUpBasic}.

Com o objetivo de acompanhar e gerir da melhor maneira possível os requisitos, foram estabelecidos os seguintes atributos de requisitos:
\begin{itemize}
\item Data de entrega:
Data em que o requisito deve ser fornecido. Os valores deste atributo serão as próprias datas planejadas de entrega do requisito.

\begin{table}[h]
\centering
\caption{Descrição dos valores do atributo Data de entrega}
\label{Rotulo}
\begin{tabular}{|l|l|}
\hline
\textbf{Valor} & \textbf{Descrição} \\
\hline
Data & Data planejada para entrega do requisito. \\ \hline
\end{tabular}
\end{table}

Esse atributo foi definido por fornecer de maneira rápida a previsão de entrega do requisito. Dessa maneira, permite uma visão ampla do planejamento de requisitos, auxiliando nas ações de gerenciamento dos requisitos.
\item Dificuldade:
Uma indicação do nível de esforço necessário ou quão difícil será implementar o requisito. Este atributo pode assumir os valores: alta, média ou baixa.

\begin{table}[h]
\centering
\caption{Descrição dos valores do atributo Dificuldade}
\label{Rotulo}
\begin{tabular}{ | l | p{1\linewidth} | }
\hline
\textbf{Valor} & \textbf{Descrição} \\ \hline
Alta  &  Requisito com implementação complexa, em função de: dificultadores técnicos, alterarem a implementação de muitos requisitos e/ou que tenham relacionamento com sistemas externos. \\ \hline
Média & Requisito com implementação não trivial, poucos dificultadores técnicos, alterarem a implementação de poucos requisitos e sem relação com sistemas externos. \\ \hline
Baixa & Requisito com uma implementação trivial, sem: dificultadores técnicos, alterações na implementação de outros requisitos ou relação com outros sistemas externos. \\ \hline
\end{tabular}
\end{table}

Esse atributo foi escolhido em razão de informar o quão difícil será implementar um requisito, fornecendo dados que auxiliam no planejamento e organização dos esforços para conclusão das atividades de desenvolvimento de requisitos e para alcançar os resultados esperados.

\item Status:
Grau de completude, ou seja, o progresso da implementação de um requisito. Este atributo pode adquirir os seguintes valores: completo, parcialmente concluído ou não Iniciado.

\begin{table}[h]
\centering
\caption{Descrição dos valores do atributo Status}
\label{Rotulo}
\begin{tabular}{ | l | p{0.5\linewidth} | }
\hline
\textbf{Valor} & \textbf{Descrição} \\ \hline
Completo &  Indica que o requisito foi completamente implementado. \\ \hline
Parcialmente concluído & Indica que o requisito foi parcialmente implementado e ainda está sendo implementado. \\ \hline
Não iniciado & Indica que o requisito ainda não começou a ser implementado. \\ \hline
\end{tabular}
\end{table}

Esse atributo foi definido com o intuito de fornecer, a qualquer momento do desenvolvimento dos requisitos, a situação atual do requisito, em relação ao andamento de sua implementação. Ele permite que se tenha uma visão total do andamento do projeto, a partir de cada requisito.

\item Prioridade:
Declaração da importância relativa do requisito para os \textit{Stakeholders} (envolvidos ou interessados no desenvolvimento do software). Pode receber os seguintes valores: alta, média ou baixa.

\begin{table}[h]
\centering
\caption{Descrição dos valores do atributo Prioridade}
\label{Rotulo}
\begin{tabular}{ | l | p{0.8\linewidth} | }
\hline
\textbf{Valor} & \textbf{Descrição} \\ \hline
Alta &  Requisitos imprescindíveis. O fracasso em sua implementação indica que o sistema não irá atender às necessidades dos interessados. São requisitos fundamentais para o sucesso do projeto. \\ \hline
Média & Requisitos importantes para a eficácia ou eficiência do sistema. Sua não implementação afeta a satisfação do usuário e/ou o valor agregado do produto e o não atendimento não determina o fracasso do projeto. \\ \hline
Baixa & Requisitos úteis, porém menos críticos, sendo usados menos frequentemente. Não possui muito significado para a satisfação do usuário e seu atendimento pode ser postergado. \\ \hline
\end{tabular}
\end{table}
\tiny Fonte:http://www.engenhariadesoftware.net.br/artigos/artigo-26-levantamento-e-gerenciamento-de-requisitos-ii

\normalsize Esse atributo foi escolhido em razão de identificar o quão importante um requisito é para os envolvidos no projeto. Ele permite priorizar e planejar a implementação dos requisitos de acordo com suas prioridades, priorizando os requisitos de prioridade mais alta.


\end{itemize}

