\section[Critérios de Avaliação das Ferramentas]{Critérios de Avaliação das Ferramentas}

De acordo com \citeonline{beatty} os critérios para avaliar uma ferramenta são: 
Armazenamento dos requisitos e de seus atributos
\begin{itemize}
  \item Artefatos que estão relacionados
  \item Gestão de mudanças
  \item Rastreabilidade dos requisitos
  \item Flexibilidade
  \item Usabilidade
  \item Adaptação ao contexto
\end{itemize} 
a partir disso, foram definidos os critérios para avaliar as ferramentas nesse trabalho.
\begin{itemize}
  \item Armazenamento dos requisitos e de seus atributos: nesses pontos, o foco da avaliação era descobrir a capacidade da ferramenta em guardar informações sobre os requisitos. Visto que há a possibilidade da mesma alterar ou simplesmente apagar dados inseridos, a partir de testes e do manual da ferramenta foi possível descobrir se com o uso da mesma, estava-se livre de tal risco.

  \item Artefatos que estão relacionados: nessa avaliação, buscou-se compreender e definir quais as relações presentes em dado requisito, ou seja, se o mesmo se relaciona com outros requisitos e se ao fim de dada iteração, algum documento é gerado.

  \item Gestão de mudanças: buscou-se avaliar o comportamento da ferramenta no que diz respeito as mudanças que o requisito pode sofrer ao longo do projeto. A importância de se avaliar isso tange a questão da qualidade do que foi proposto, se de fato é pouco mutável e menos sucetível a falhas.

  \item Rastreabilidade dos requisitos: buscou-se a avaliação para saber se a rastreabilidade dos requisitos é realizada de maneira satisfatória, fornecendo características essenciais como origem dos requisitos em seus diferentes níveis de abstração e dependências entre os requisitos.

  \item Flexibilidade: neste critério, buscou-se avaliar a capacidade que a ferramenta tem de se adequar aos projetos. Se a ferramenta suporta desde um projeto menor até um projeto mais complexo.

  \item Usabilidade: com essa questão, buscou-se avaliar as facilidades encontradas no manuseio da ferramenta, se ela possuia nomes significativos, se a mesma é executada de maneira fluída, sem muitos travamentos, se a interface era fácil de usar e se era fácil de se lembrar como se usa.
  
  \item Adaptação ao contexto: nesse ponto, fez-se uma verificação e análise a fim de descobrir se a ferramenta oferecia um suporte considerável e 
  único para a abordagem definida pelo time e para o processo definido.
\end{itemize}

Após a pesquisa das ferramentas para gestão de requisitos e a análise de cada uma das encontradas, foi feito um quadro 
comparativo com critérios definidos acima, que pode ser visto na Figura \ref{fig:quadro} que avalia cada uma das ferramentas. Cada um desses critérios foram avaliados de 0 a 5, sendo:

0 - Muito ruim - Não atende de forma alguma ao critério.

1 - Ruim - Não atende ao critério da forma esperada.

2 - Razoável - Atende ao critério, mas seu retorno não é de forma significativa.

3 - Bom - Atende ao critério da forma esperada.

4 - Muito bom - Atende ao critério além da forma esperada.

5 - Excelente - Atende ao critério além da forma esperada, com um retorno bastante significativo e de qualidade.

\begin{table*}[!h]
\caption{Comparação entre ferramentas}
\label{fig:quadro}
\begin{tabular}{p{0.45\linewidth}p{0.15\linewidth}p{0.15\linewidth}p{0.15\linewidth}}
  \hline
  Item/Ferramenta & Tiger Pro & TraceCloud & Caliber \\
  \hline			
  Armazenamento dos requisitos & 3 & 4 & 4\\
  \hline
  "Armazenamento de atributos dos requisitos"	& 2 & 4 & 4\\
  \hline
    "Armazenamento de artefatos relacionados aos requisitos"	& 1 & 3 & 4\\
  \hline
      "Armazenamento de artefatos relacionados aos requisitos"	& 1 & 3 & 4\\
  \hline
       "Relacionamento entre os requisitos"	& 2 & 5 & 4\\
  \hline
  Gestão de mudanças	&3	&3&	3\\
  \hline
  Rastreabilidade	&3&	5	&4\\
  \hline
  Flexibilidade	&2&	3	&3\\
  \hline
  Usabilidade 	&3	&5	&3\\
  \hline
  "Para o processo de Engenharia de Requisitos da Empresa Junior de Engenharia de Energia"	&2	&5	&5\\
  \hline
    Tipo &Gratuito	&Gratuito por 60 dias	&Gratuito por 30 dias\\
    
  \hline
  \end{tabular}
\end{table*}




