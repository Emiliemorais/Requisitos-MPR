\section[Rastreabilidade de Requisitos]{Rastreabilidade de Requisitos}

A rastreabilidade pode ser dividida em pré-rastreabilidade e pós-rastreabilidade. 
A pré-rastreabilidade consiste no rastreamento de informações anteriores a especificação
de requisitos. Como por exemplo, a fonte dos requisitos que podem ser: \textit{stakeholders},
documentos e regras de negócio. Já a pós-rastreabilidade está relacionada com os requisitos depois 
de serem especificados e está mais focada na solução. \cite{persson}

A rastreabilidade também pode ser horizontal ou vertical. A horizontal está relacionada com
as informações de uma mesma fase do processo de desenvolvimento e a vertical
com informações entre várias fases do processo. \cite{persson}

De acordo com Paetsch, Eberlein e Maurer (\citeyear{paetsch}), a rastreabilidade auxilia na gerência de mudanças, 
pois estabelece um relacionamento entre os requisitos, o projeto e a implementação do sistema. Nuseibeh e Easterbrook (\citeyear{nuseibeh})
afirmam que a rastreabilidade é o coração do gerenciamento de requisitos.

No SAFe os requisitos possuem três níveis diferentes de abstração: tema de investimento, épicos, \textit{features} e histórias.
A partir disso, a rastreabilidade definida para esse trabalho pode ser vista na Figura \ref{fig:rastreabilidade}.
Será realizada uma rastreabilidade, através da ferramenta de gerenciamento de requisitos, vertical e horizontal. Assim acompanhando
a origem das histórias, ou seja, de qual \textit{feature} e épico é derivada e também monitorar as dependências entre as histórias,\textit{features} e épicos.

Também será realizada a pré-rastreabilidade para identificar a fonte dos requisitos e a pós-rastreabilidade que irá rastrear a história que um determinado
código implementa. A pós-rastreabilidade será feita apenas para algumas histórias que serão priorizadas e desenvolvidas nesse trabalho.
Essa rastreabilidade será feita através da ferramenta de gerenciamento de requisitos.

\graphicspath{{figuras/}}

\begin{figure}[!htb]
 \centering
 \includegraphics[width = 18cm, height = 15cm]{rastreabilidade}
 \caption{Estratégia de rastreabilidade}
 \label{fig:rastreabilidade}

\end{figure}