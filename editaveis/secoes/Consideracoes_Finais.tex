\chapter[Considerações Finais]{Considerações Finais}

\section{Relato de experiência da disciplina de Engenharia de Requisitos de Software}

\indent A disciplina é de extrema importância para a formação de um estudante de Engenharia de Software e a 
forma com que a mesma é ministrada reflete em uma experiência valiosa adquirida pelo aluno, pois este, ao curso 
da disciplina, está lidando com situações reais de negócio que ocorrem no mercado de trabalho. É a primeira vez 
que o aluno trabalha com clientes reais, que possui um problema a ser solucionado e isso causa uma evolução na 
linha de pensamento e a abstração dos alunos.

\indent Em um primeiro momento, a falta de maturidade e fixação de conceitos referentes a Engenharia de Requisitos 
gerou insegurança por parte dos alunos na realização de tarefas referentes ao trabalho que com o passar do tempo 
foram se reduzindo.

\indent Um ponto interessante que pode ser percebido na execução do projeto da disciplina, é a necessidade de definir 
uma abordagem ou como o trabalho para extração ou elicitação dos requisitos vai ser realizado, o que deixa o processo 
menos abstrato e melhora a visibilidade das ações que estão sendo feitas no processo.

\indent A disciplina, em sua atual maneira de ser ministrada, possui uma interação com outra disciplina. A interação 
com o grupo da disciplina de Melhoria de Processos foi importante pelo fato de simular como ocorre no mercado de 
trabalho a relação entre o cliente e os profissionais da área de Engenharia de Requisitos. Nesse grupo em especial, 
pelo fato haver integrantes de outras engenharias, foi interessante aprender a lidar com visões completamente opostas 
assim como rotinas e vocabulários diferentes. Uma experiência que além de aprender, o escopo do projeto, pode ter ajudado 
a desenvolver habilidades como oratória, escrita e práticas de como lidar com pessoas envolvidas em outro contexto 
educacional.

\indent Um ponto positivo foi a formação dos grupos para o projeto da disciplina. Esta formação foi feita através de uma 
dinâmica e esta não permitia com que os integrantes formassem seus grupos de forma direta, o que proporcionou ao aluno uma 
aprendizagem em trabalhar em equipe da melhor forma possível, e formar equipe com alunos a qual você nunca trabalhou antes 
faz com que o aluno aprenda a lidar com todo tipo de pessoa no mercado de trabalho.


\indent Outro ponto positivo foi a maneira com que o professor ministra suas aulas são interessantes e ele traz diversos 
exemplos como demonstração para que fique claro o conteúdo. O professor possui uma relação muito boa com os alunos, sempre
disposto a ajudar e tirar dúvidas. Suas críticas são bastante construtivas e, por mais que pareça muito exigente, ele está 
apenas nos tornando capazes de enfrentar todas as barreiras existentes no mercado de trabalho com bastante profissionalismo 
e ética.

\indent Um ponto negativo foi a mudança no cronograma do curso da disciplina, pois essas mudanças afetaram no andamento do 
trabalho proposto gerando perda de algumas aulas importantes.

\indent A forma com que a disciplina é ministrada exige bastante esforço do aluno, pois ele tem que colocar em prática em 
pouco tempo o que está estudando e dessa forma agregando bastante valor em sua vida pessoal e profissional.


\section{Relato de experiência da execução do trabalho}

\indent Apesar das dificuldades iniciais de compreensão e definição de uma abordagem, a equipe acredita que as técnicas e 
práticas escolhidas tenham sido adequadas para o desenvolvimento do projeto devido principalmente à formalização e ao 
dinamismo das operações do cliente, visto que o projeto atende a uma empresa nova e sem rigidez ou definição clara de seus 
processos. Em razão disso e de outras características da empresa, do contexto e da equipe, utilizamos uma abordagem adaptativa 
e, devido à dinâmica imposta pela disciplina e pela abordagem escolhida, os resultados e a experiência obtida perante o 
desenvolvimento do projeto foram positivos, fazendo disso uma recomendação de uso da mesma e reutilização da abordagem em 
outros projetos por parte da equipe.

\indent Entretanto, houve algumas dificuldades em relação a como executar algumas atividades específicas da abordagem, pois 
interferiram no resultado das atividades diminuindo a produtividade e aumentando o tempo de execução delas.

\indent Além da escolha da abordagem, foi necessário fazer outras escolhas no momento do planejamento. Essas escolhas foram 
feitas com base em pesquisas, todavia era algo totalmente novo para o grupo. Das escolhas realizadas, obteve-se êxito, além 
da abordagem já citada acima, na definição da ferramenta, e de algumas das técnicas de elicitação de requisitos. Uma das 
técnicas escolhidas, a análise documental, mostrou-se inutilizada na execução do que havia sido planejado, pois o contexto 
de negócio impediu a eficácia dessa execução devido a ser uma empresa nova, sem experiência e com pouca documentação 
consolidada.

\indent Apesar das dificuldades expostas acima, elas não impediram a realização das mesmas, e chegamos a um resultado satisfatório 
ao final da disciplina. Além disso, ao passo que tivemos essas dificuldades e fomos alinhando o conhecimento, ajustamos o processo 
e as definições de projeto de engenharia de requisitos para adequá-los aos conceitos e as práticas corretas da disciplina, com base 
nas aulas, orientações do professor ou referência bibliográfica.


\indent A realização do trabalho foi uma oportunidade de grande amadurecimento, pelo fato da experiência de montagem de um organograma 
de processo repleto de atividades a serem tratadas para solucionar determinado problema.