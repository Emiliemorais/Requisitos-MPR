\chapter[Considerações Finais]{Considerações Finais}
% 
% Perpassadas algumas etapas de busca e construção de conhecimentos sobre os Requisitos, além de estudos sobre a definição de um processo de Engenharia de Requisitos e temas correlacionados ao desenvolvimento de \textit{software}, pôde-se observar o quão ampla é a área de Requisitos, e como vem se aperfeiçoando a cada dia.
% 
% Com o objetivo de definir um processo de Engenharia de Requisitos para ser aplicado ao contexto da Empresa Júnior Matriz, foi possível aplicar conhecimentos, metodologias, técnicas e análises na prática, durante o decurso de definição do processo Engenharia de Requisitos. O que possibilitou aos membros das equipes, tanto de Requisitos quanto de Modelagem, uma ganho valioso de conhecimento. Destaca-se aqui, a equipe de Requisitos que trabalhou de forma integrada e esforçada para executar as tarefas e atender aos prazos definidos.
% 
% Apesar do ganho de conhecimento, também existiram dificuldades na compreensão de alguns conceitos e técnicas na área de Requisitos, além de contratempos na 
% organização do grupo, definição de reuniões e outros inconvenientes, inerentes ao trabalho em equipe. 
% No entanto, por meio do empenho e dedicação dos integrantes da equipe foi possível vencer esses contratempos e inconvenientes.
% 
% Espera-se que o processo de Engenharia de Requisitos definido possa ser aplicado de forma a alcançar os objetivos da Empresa Matriz, agregando valor e 
% resolvendo uma questão problemática da empresa. Além de permitir aos membros da equipe, a vivência na prática do conteúdo visto em sala de aula. 