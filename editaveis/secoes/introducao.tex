\chapter[Introdução]{Introdução}

Segundo \citeonline{pressman}, Engenharia de Requisitos é ``um amplo
espectro de tarefas e técnicas que levam a um entendimento dos requisitos''.
\citeonline{jitnah} afirmam que é o processo através do qual as necessidades dos usuários 
são identificadas e expressas em um artefato.

A partir disso, é necessário estudar o contexto de negócio, definir técnicas de elicitação, 
definir estratégias de gerenciamento de requisitos, definir atividades, artefatos, papéis do processo
e definir uma ferramenta que dê suporte ao processo e gerenciamento definidos.
Esses estudos foram realizados e a partir disso foi definido um processo de Engenharia de Requisitos para uma empresa júnior, 
chamada Matriz Engenharia de Energia. Com o estudo do contexto de negócio e através da análise de perfil do cliente e da equipe, o processo
estabelecido possui uma abordagem adaptativa baseada nos \textit{frameworks} SAFe e Scrum. 

Esse trabalho relata a execução desse processo definido anteriormente, dessa forma apresentando os requisitos obtidos e 
a experiência com o uso das técnicas e ferramenta escolhidas.

\section{Organização do Trabalho}

Este relatório está organizado nos seguintes capítulos: Contexto de Negócio, Processo de Engenharia de Requisitos, Técnicas de Elicitação,
Nível de Portfólio, Nível de Programa, Nível de Time, Tópicos de Gerenciamento de Requisitos e Considerações Finais.

No \textit{Capítulo 2: Contexto de Negócio} é apresentado o contexto no qual o processo foi executado.

No \textit{Capítulo 3: Processo de Engenharia de Requisitos} é apresentado o modelo do processo de Engenharia de Requisitos.

No \textit{Capítulo 4: Técnicas de Elicitação} são apresentadas as técnicas de elicitação que foram utilizadas.

No \textit{Capítulo 5: Nível de Portfólio} são apresentados os requisitos identificados nesse nível.

No \textit{Capítulo 6: Nível de Programa} é apresentado o documento de visão do sistema, o \textit{Roadmap} e os requisitos identificados
nesse nível.

No \textit{Capítulo 7: Nível de Time} é apresentado o planejamento da iteração e as histórias especificadas.

No \textit{Capítulo 8: Tópico de Gerenciamento de Requisitos} são apresentados os atributos dos requisitos e a rastreabilidade dos requisitos.

No \textit{Capítulo 9: Considerações Finais} é apresentada a conclusão obtida
com este relatório.