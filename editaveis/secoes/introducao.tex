\chapter[Introdução]{Introdução}

Segundo Pressman (\citeyear{pressman}), Engenharia de Requisitos é ``um amplo
espectro de tarefas e técnicas que levam a um entendimento dos requisitos.''.
Jitnah, Han e Steele (\citeyear{jitnah}) afirmam que é o processo através do qual as necessidades dos usuários 
são identificadas e expressas em um artefato.

A partir disso, é necessário estudar o contexto de negócio, definir técnicas de elicitação, 
definir estratégias de gerenciamento de requisitos, definir atividades, artefatos, papéis do processo
e definir uma ferramenta que dê suporte ao processo e gerenciamento definidos.

Dessa forma, este trabalho tem como objetivo definir um processo de Engenharia de Requisitos para uma empresa júnior, chamada
Matriz Engenharia de Energia, com o intuito de solucionar um problema da organização por meio de um sistema de software,
realizando todas as atividades supracitadas.

Com o estudo do contexto de negócio e através da análise de perfil do cliente e da equipe, o processo
estabelecido neste trabalho possui uma abordagem adaptativa baseada nos \textit{frameworks} SAFe e Scrum.

\section{Organização do Trabalho}

Este relatório está organizado nos seguintes capítulos Contexto de negócio, Justificativa da Abordagem, 
Processo de Engenharia de Requisitos, Elicitação, Tópico de Gerenciamento de Requisitos,
Planejamento, Ferramentas de Gestão de Requisitos e Considerações Finais.

No \textit{Capítulo 2: Contexto de negócio} é apresentado o contexto no qual o processo
definido irá atuar.

No \textit{Capítulo 3: Justificativa da Abordagem} são apresentados argumentos
que implicaram na escolha da abordagem adaptativa para este trabalho.

No \textit{Capítulo 4: Processo de Engenharia de Requisitos} é descrito o processo de Engenharia
de Requisitos definido com suas atividades, tarefas, papéis e artefatos.

No \textit{Capítulo 5: Elicitação} são apresentadas as técnicas de elicitação
a serem utilizadas na execução do processo definido.

No \textit{Capítulo 6: Tópico de Gerenciamento de Requisitos} são abordados
os atributos de requisitos e a estratégia de rastreabilidade, para gerenciamento
dos requisitos.

No \textit{Capítulo 7: Planejamento} é apresentado um cronograma de atividades
definido para guiar o andamento das atividades.

No \textit{Capítulo 8: Ferramentas de Gerenciamento de Requisitos} são apresentadas
algumas ferramentas de gerenciamento de requisitos selecionadas para avaliação. 
A avaliação e a escolha de uma ferramenta também é abordada nesse capítulo.

No \textit{Capítulo 9: Considerações Finais} é apresentada a conclusão obtida
com este relatório e os resultados esperados com o próximo trabalho.