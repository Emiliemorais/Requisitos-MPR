\chapter[Introdução]{Introdução}

Segundo \citeonline{pressman}, Engenharia de Requisitos é ``um amplo
espectro de tarefas e técnicas que levam a um entendimento dos requisitos''.
\citeonline{jitnah} afirmam que é o processo através do qual as necessidades dos usuários 
são identificadas e expressas em um artefato.

A partir disso, é necessário estudar o contexto de negócio, definir técnicas de elicitação, 
definir estratégias de gerenciamento de requisitos, definir atividades, artefatos, papéis do processo
e definir uma ferramenta que dê suporte ao processo e gerenciamento definidos.
Esses estudos foram realizados e a partir disso foi definido um processo de Engenharia de Requisitos para uma empresa júnior, 
chamada Matriz Engenharia de Energia, com o intuito de solucionar um problema da organização por meio de um sistema de software,
realizando todas as atividades supracitadas.

Com o estudo do contexto de negócio e através da análise de perfil do cliente e da equipe, o processo
estabelecido neste trabalho possui uma abordagem adaptativa baseada nos \textit{frameworks} SAFe e Scrum.

\section{Organização do Trabalho}

Este relatório está organizado nos seguintes capítulos: Contexto de Negócio...

No \textit{Capítulo 2: Contexto de Negócio} 

No \textit{Capítulo 3: Identificação dos Requisitos} 

No \textit{Capítulo 4: Técnicas de Elicitação} 

No \textit{Capítulo 5: Histórias de Usuário} 

No \textit{Capítulo 6: Tópico de Gerenciamento de Requisitos} 

No \textit{Capítulo 9: Considerações Finais} é apresentada a conclusão obtida
com este relatório.