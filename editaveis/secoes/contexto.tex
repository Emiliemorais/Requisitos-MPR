\chapter[Contexto da Empresa]{Contexto da Empresa}
O contexto proposto, se refere a empresa júnior Matriz - Engenharia de Energia, que é 
formada por alunos da Engenharia de Energia da Universidade de Brasilia, Campus Gama. 
O objetivo principal da Matriz é a iniciação dos alunos do curso no mercado de trabalho 
fornecendo aos clientes soluções inteligentes referentes a energia e sustentabilidade.

A empresa visa atender a demanda de residências, condomínios, empresas de médio 
e pequeno porte. A Matriz, tenta desenvolver serviços exclusivos como estudos de
viabilidade energética, projetos de geração distribuída de energia, adequação tarifária, 
controle e gestão inteligente de energia, correção de fator de potência, auxílio e suporte
em comercialização de energia, desenvolvimento de projetos elétricos além do oferecimento 
de cursos e treinamentos voltados à capacitação.

\section{O Problema}
Como já dito, a Matriz vêm com a proposta de ser uma Empresa Júnior que proporcione para 
os alunos de graduação uma experiência empresarial. Por ser nova, a empresa não possui 
um processo bem definido que resulte na aquisição de novos clientes. Inicialmente o que 
foi definido, consistia em investir no uso de redes sociais para difundir suas ideias, 
no estudo de mercado no que tange empresas que mais gastam energia e enviar via telefone
propostas dos serviços que a empresa disponibiliza. Feito isso, optou-se então pelo uso de
cartazes nos corredores da Faculdade do Gama (FGA), porém nenhuma das estratégias seguidas
surtiu efeito no que diz respeito a adesão de sócios e clientes.

\section{Possível Solução}
A partir deste quadro, o grupo da disciplina de Requisitos de Software, junto com um 
grupo da disciplina de Modelagem de Processos, foi encubido de criar estratégias 
criativas para obter uma maior adesão a Matriz. O problema será solucionado 
usando conceitos referentes a Engenharia de Requisitos, juntos a metodologias de 
desenvolvimento adaptativas, desenhando assim um processo que consistirá desde o nível 
de análise do que de fato se necessita para reverter esse quadro, até o ponto de 
apresentar uma solução em software para o problema proposto.

Serão realizadas portanto, interações com os interessados no produto final para que 
os mesmos possam fornecer dados que possibilitem além de um auxílio aos responsáveis 
por desenvolver o projeto, negociações e documentações de possíveis planos que serão
mudados.
