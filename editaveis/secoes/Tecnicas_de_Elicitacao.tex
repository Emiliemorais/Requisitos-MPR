\chapter[Técnicas de Elicitacao]{Técnicas de Elicitação de Requisitos}

A elicitação é uma das atividades fundamentais da Engenharia de Requisitos e consiste no processo de identificar
itens de informação que determinam as características de um sistema. \cite{jitnah} 

Para \citeonline{paetsch} a elicitação de requisitos tem o intuito de descobrir
os requisitos e identificar as fronteiras do sistema, estas definem o contexto, consultando os \textit{stakeholders} (cliente, usuários e desenvolvedores).

A atividade de elicitação pode ser realizada através da aplicação de várias técnicas, como: Grupos focais, entrevistas, questionários,
introspecção, análise de protocolo, prototipagem, animação, análise de cenário, estudo etnográfico, observação, análise de tarefas,
\textit{workshops} e \textit{brainstorming}. (\cite{jitnah} ; \cite{coulin})

\label{tecnicas}

As técnicas de elicitação de requisitos foram selecionadas de acordo com a interação entre a equipe e o cliente e de acordo com os níveis de conhecimento e de abstração dos requisitos a serem obtidos.
As técnicas escolhidas foram:\\
\begin{itemize}
\item Análise documental;
\item \textit{Workshop} de Requisitos;
\item Entrevistas.
\end{itemize}

\section{Análise Documental}
Por ser uma empresa nova e não ter executado o processo de captação de clientes o único documento analisado foi o Plano
de negócios da Matriz. Dessa forma, foi perceptível que a escolha desta técnica foi errônea, pois não levou
em consideração a imaturidade da empresa e a falta de documentos como formulários ou fichas que poderiam ter auxiliado
na elicitação de alguns requisitos.

\section{\textit{Workshop}}
  Foram realizados dois \textit{Workshops} com o intuito de Estabelecer o Tema de Investimento e Levantar Épicos. Nos \textit{Workshops}
  foram definidos papéis para organizar de forma efetiva a reunião. Os papéis e as responsabilidades podem ser vistos abaixo.
  
\textbf{Facilitador:}

  \begin{itemize}
	\item Dirige o \textit{Workshop};
	\item Deixa bem claro o objetivo de cada passo do \textit{Workshop};
	\item Não permite críticas ou debates durante o \textit{Workshop}.
	\end{itemize}

\textbf{Moderador:}

  \begin{itemize}
	\item Controle do tempo;
	\item Manter foco do \textit{Workshop}.
	\end{itemize}

\textbf{Registrador:}
  
  \begin{itemize}
	\item Registra o que teve de importante no \textit{Workshop}.
	\end{itemize}
O primeiro \textit{Workshop} ocorreu com base no planejamento da Tabela \ref{fig:workshop1} e o seu objetivo
era estabelecer tema de investimento e iniciar o levantamento dos épicos. Todavia, houve uma dificuldade no entendimento
do problema da empresa, dessa forma sendo necessário a execução de outro \textit{Workshop}. 
Assim, o segundo \textit{Workshop} foi realizado com o intuito de estabelecer o tema de investimento e levantar os épicos e aconteceu conforme o planejamento da Tabela
\ref{fig:workshop2}. Ambos os planejamentos foram feitos utilizando o exemplo de \citeonline{safe}.

\begin{table*}[!h]
\centering
\caption{Planejamento do \textit{Workshop} 1}
\label{fig:workshop1}
  \begin{tabular}{|p{0.20\linewidth}|p{0.25\linewidth}|p{0.40\linewidth}|}
  \hline
   Hora  & Atividade & Descrição\\
  \hline

  17:00 a 17:10 & Apresentação & Os participantes se apresentam e o time diz seus papéis para o \textit{Workshop}\\ \hline

  17:10 a 17:30  & Contextualização & O contexto e o modelo de negócio são apresentados pelos \textit{Product Managers} (Equipe de Modelagem)\\\hline

  17:30 a 17:45 & Estabelecimento do Tema de Investimento & O tema de investimento é escrito para que todos possam visualizar e é estabelecido e validado\\\hline
  
  17:45 a 18:00 & Exposição de ideias & As ideias de todos os \textit{Product Managers} são expostas de modo informal no quadro branco, para esboçar requisitos de alto nível\\\hline

  18:00 a 18:30 & Discussão das ideias apresentadas & As ideias expostas são discutidas de modo a serem analisadas para verificar se correspondem realmente ao desejado\\\hline

  \hline
  \end{tabular}
\end{table*}
\vfill
\pagebreak

\begin{table*}[!h]
\centering
\caption{Planejamento do \textit{Workshop} 2}
\label{fig:workshop2}
  \begin{tabular}{|p{0.20\linewidth}|p{0.25\linewidth}|p{0.40\linewidth}|}
  \hline
  Hora  & Atividade & Descrição\\
  \hline

16:10 a 16:20  & Apresentação & É apresentado os objetivos do \textit{Workshop} e a divisão dos papéis \\\hline

16:20 a 16:50 & Exposição de ideias & As ideias de todos os \textit{Product Managers} são expostas de modo informal no quadro branco, para esboçar requisitos de alto nível \\\hline

16:50 a 17:00 & Estabelecimento do Tema de Investimento & O tema de investimento é escrito para que todos possam visualizar e é estabelecido e validado\\\hline

17:00 a 17:10 & Escrita dos épicos & Os épicos são escritos a partir das ideias esboçadas \\\hline

17:10 a 17:25 & Validação dos épicos & Os épicos escritos são validados pelo \textit{Product Managers}\\\hline

  \hline
  \end{tabular}
\end{table*}

\section{Entrevista}

A partir dos épicos levantados, as entrevistas foram utilizadas nas atividades de Levantar \textit{Features} e Identificar Requisitos Não Funcionais.
As entrevistas foram mistas, ou seja, algumas perguntas estavam prontas e no momento da reunião com o cliente surgiram
outras perguntas. As questões utilizadas encontram-se no Apêndice \ref{questionario}.

Essa técnica foi bastante útil, pois permitiu explorar pontos do sistema de forma que o cliente conseguisse
expressar suas necessidades e desejos e a partir disso foi possível definir as \textit{features} e os requisitos não funcionais.

%Escrever mais sobre como foi usar a técnica de entrevista
