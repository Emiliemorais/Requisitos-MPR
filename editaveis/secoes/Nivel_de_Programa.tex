\chapter[Nível de Programa]{Nível de Programa}

\section{Requisitos Identificados}

\textbf{\textit{Feature} 01 - Listagem de contato de possíveis clientes}

Nesta \textit{feature}, espera-se obter uma lista de contatos oriúnda de registros, fornecendo portanto a oportunidade de edição e visualização de todos os contatos, assim como seus respectivos atributos e desejos.


\textbf{\textit{Feature} 02 - Padronização da divulgação dos serviços}

A partir de tal \textit{feature}, há o objetivo de incluir um padrão de apresentações para melhor abordar os clientes. Com isso, um ponto a ser agregado é a forma com que os emails serão enviados além de fornecer a visualização dos serviços que serão oferecidos pela empresa no presente período de tempo.


\textbf{\textit{Feature} 03 - Acompanhamento do contato com o cliente}

O objetivo da \textit{feature}, é trabalhar com a criação de círculos para com o cliente. Isso vai envolver o oferecimento dos serviços e de uma cartilha com as propostas. Além desse cenário favorável, deve-se registrar os clientes que rejeitaram a proposta em um primeiro momento e anotar seu \textit{feedback}, caso haja.


\textbf{\textit{Feature} 04 - Acompanhamento das reuniões com o cliente}

Nesta \textit{feature}, espera-se fazer todo o tratamento que envolve as reuniões, desde o planejamento da mesma, no que se refere as datas e locais das mesmas assim como a pauta do que será falado, até uma geração de relatórios do que foi debatido na mesma.		


\textbf{\textit{Feature} 05 - Controle das informações referentes ao projeto}

Se busca obter a partir de relatórios, o que de fato esta sendo realizado no projeto, se esta sendo feita da melhor maneira. Há opções capazes de realizar registros de novos projetos, custos, pessoas envolvidas, avaliar se o projeto é de fato viável etc.

Uma vez que as \textit{features} foram identificadas e especificadas, sabe-se que cada uma possui algumas histórias de usuário, que nada mais são do que pequenas descrições que se espera que a aplicação forneça. Abaixo estão listadas as histórias coletadas. Para melhor organização, optou-se por usar o termo US, que advêm de \textit{User Story}.

\textbf{Histórias}

US-01 Eu como membro da equipe de \textit{marketing} gostaria de registrar os clientes e seus contatos que foram encontrados durante a pesquisa para obter de forma organizada seus dados


US-02 Eu como membro da equipe de \textit{marketing} gostaria de registrar o detalhamento das informações acerca do cliente já cadastrado para poder ter uma visibilidade do perfil do meu possível cliente


US-03 Eu como membro da equipe de \textit{marketing} consultor de \textit{marketing} gostaria de editar os clientes e seus contatos que foram encontrados durante a pesquisa para manter suas informações atualizadas


US-04 Eu como membro da equipe de \textit{marketing} gostaria de editar o detalhamento das informações acerca do cliente já cadastrado para manter o registro atualizado


US-05 Eu como membro da equipe de \textit{marketing} gostaria de consultar os clientes registrados para saber informações de um cliente específico


US-06 Eu como diretor de \textit{marketing} gostaria de visualizar um relatório contendo os clientes, seus contatos e perfis para entrar em contato com ele e oferecer os serviços da Matriz


US-07 Eu como diretor de \textit{marketing} gostaria de armazenar um modelo de apresentação para padronizar a apresentação da reunião


US-08 Eu como diretor de \textit{marketing} gostaria de armazenar uma cartilha informativa para poder padronizar o conteúdo do e-mail a ser enviado para o cliente


US-09 Eu como diretor de \textit{marketing} gostaria de registrar os serviços oferecidos pela Matriz para poder oferecê-los aos clientes durante o contato


US-10 Eu como funcionário gostaria de consultar os serviços oferecidos pela Matriz para conhecer os serviços cadastrados


US-11 Eu como diretor de \textit{marketing} gostaria de editar os serviços oferecidos pela Matriz para poder manter os serviços atualizados


US-12 Eu como funcionário gostaria de visualizar os clientes que não manifestaram interesse nos serviços da Matriz para posteriormente entrar em contato novamente


US-13 Eu como diretor de \textit{marketing} gostaria de registrar um contato realizado com o cliente para manter o controle dos clientes já contatados


US-14 Eu como diretor de \textit{marketing} gostaria de enviar uma cartilha informativa ao meu cliente para tentar angariar sua fidelidade


US-15 Eu como membro da equipe \textit{marketing} gostaria de agendar uma reunião com meu cliente para manter a organização de minha agenda de reuniões


US-16 Eu como membro da diretoria gostaria de imprimir um relatório estátistico do percentual de clientes captados para tomar decisões estratégicas acerca do \textit{marketing}


US-17 Eu como diretor de \textit{marketing} gostaria de editar o registro de um contato realizado com o cliente para manter o registro do contato atualizado


US-18 Eu como membro da diretoria gostaria de acessar a apresentação do slide para conseguir fazer a apresentação durante a reunião com o cliente da Matriz


US-19 Eu como funcionário gostaria de visualizar a agenda de reuniões de clientes para poder ter conhecimento acerca dos dias e horários das reuniões


US-20 Eu como membro do \textit{marketing} gostaria de remarcar uma reunião com meu cliente para atualizar a organização de minha agenda de reuniões


US-21 Eu como membro da diretoria gostaria de cadastrar as anotações da reunião para posteriormente verificar a veracidade das informações ditas pelo cliente com os reais problemas existentes


US-22 Eu como membro da diretoria gostaria de consultar as anotações da reunião para verificar a veracidade das informações ditas pelo cliente com os reais problemas existentes


US-23 Eu como membro da equipe de projeto gostaria de registrar agenda de uma visita técnica para poder acompanhar as datas de todas as visitas técnicas agendadas


US-24 Eu como membro da equipe de projeto gostaria de editar a agenda de uma visita técnica para manter a agenda atualizada


US-25 Eu como membro da equipe de projeto gostaria de consultar a agenda de uma visita técnica para acompanhar saber a data e horário da visita


US-26 Eu como membro da equipe de projeto gostaria de excluir a agenda de uma visita técnica para manter a agenda atualizada


US-27 Eu como membro da equipe de projeto gostaria de cadastrar as informações da visita técnica realizada com meu cliente para escrever o pré-projeto posteriormente


US-28 Eu como membro da equipe de projeto gostaria de consultar as informações da visita técnica realizada com meu cliente para conhecer sobre a visita técnica


US-29 Eu como membro da equipe de projeto gostaria de registrar comentários sobre as anotações da visita técnica para incluir informações adicionais


US-30 Eu como gerente de projeto gostaria de registrar as informações acerca do pré-projeto do meu cliente para servir como proposta de serviço ao cliente


US-31 Eu como gerente de projeto gostaria de editar as informações acerca do pré-projeto do meu cliente para manter o pré-projeto atualizado


US-32 Eu como membros da diretoria gostaria de consultar as informações acerca do pré-projeto do meu cliente para acompanhar as informações sobre o pré-projeto


US-33 Eu como presidente gostaria de consultar um relatório de acompanhamento de todos os projetos dos clientes da Matriz para tomar decisões estratégicas


US-34 Eu como membro do setor financeiro gostaria de registrar os preços dos serviços similares para aumentar o grau de precisão na avaliação do pré-projeto


US-35 Eu como membro do setor financeiro gostaria de editar os preços dos serviços similares para manter os preços atualizados com o mercado


US-36 Eu como membro do setor financeiro gostaria de consultar os preços de serviços similares para calcular os custos do pré-projeto


US-37 Eu como membros da diretoria gostaria de visualizar o relatório de clientes com projetos aprovados pelo presidente para tomar decisões estratégicas


US-38 Eu como membros da diretoria gostaria de imprimir um contrato de prestação de serviço para registrar formalmente a contratação dos serviços da Matriz ao cliente


US-39 Eu como membros da diretoria gostaria de visualizar o relatório contendo o total de custos em projeto para auxiliar na contabilidade dos recursos da Matriz


US-40 Eu como funcionário gostaria de acompanhar o status de um projeto ao longo do processo para ter o controle do andamento do projeto


\section{Visão}

\section{\textit{Roadmap}}

O \textit{Roadmap} definido juntamente com o PM encontra-se na Figura \ref{roadmap}.

\begin{figure}[!htb]
\centering
\includegraphics[scale=0.6]{figuras/roadmap.png}
\caption{\textit{Roadmap} do projeto}
\label{roadmap}
\end{figure}

\section{Planejamento da \textit{Release}}