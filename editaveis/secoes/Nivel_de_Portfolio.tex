\chapter[Nível de Portfólio]{Nível de Portfólio}

No processo executado esse nível visa levantar os requisitos de mais altos nível, relacionados ao negócio. Esse levantamento
ocorreu através da execução das atividades Estabelecer o Tema de Investimento e Levantar Épicos.

\section{Requisitos Identificados}

\textbf{Tema de Investimento - \textit{Marketing}}

O tema de investimento definido pelo cliente foi \textit{Marketing}, dado que a intenção da empresa era investir na captação de clientes que está
dentro do processo de \textit{Marketing}.

\textbf{Épico 01 - Divulgação do Serviço}

Esse épico, têm como objetivo descrever o processo de contato inicial estabelecido com os clientes. É usado desde um nível mais primário que envolve a aquisição dos contatos e geração de uma lista de possíveis clientes, até o registro do resultado da interação realizada junto ao contato.


\textbf{Épico 02 - Acompanhamento do Processo}

Tal épico, visa uma melhor forma de visualizar a forma que o processo esta sendo executado, se as atividades estão obtendo os resultados esperados, possíveis melhorias e pontos a serem refatorados, além de fornecer relatórios referentes a reuniões com clientes.
