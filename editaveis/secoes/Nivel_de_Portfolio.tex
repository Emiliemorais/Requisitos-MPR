\chapter[Nível de Portfólio]{Nível de Portfólio}

No processo executado esse nível visa levantar os requisitos de mais alto nível, relacionados ao negócio. Esse levantamento
ocorreu através da execução das atividades Estabelecer o Tema de Investimento e Levantar Épicos, nas quais foram obtidas via análise documental e \textit{workshops}.

\section{Requisitos Identificados}

\textbf{Tema de Investimento:} TM-01 \textit{Marketing}

Este foi o tema de investimento definido pelo cliente, dado que a intenção da empresa era investir na captação de clientes que está
dentro do processo de \textit{Marketing}.

\textbf{Épico 01:} EP-01 Divulgação do Serviço

Tem como objetivo descrever o processo de contato inicial estabelecido com os clientes. É usado desde um nível mais primário que envolve a aquisição dos contatos e geração de uma lista de possíveis clientes, até o registro do resultado da interação realizada junto ao contato.


\textbf{Épico 02:} EP-02 Acompanhamento do Processo

Espera-se obter uma melhor visualização da forma que o processo está sendo executado, se as atividades estão obtendo os resultados esperados, possíveis melhorias e pontos a serem refatorados, além de fornecer relatórios referentes as reuniões com clientes.
