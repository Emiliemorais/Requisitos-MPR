\chapter[Justificativa]{Justificativa da Abordagem}



A abordagem definida foi a adaptativa com o uso dos \textit{frameworks} \emph{Scaled Agile Framework} (SAFe) e Scrum. 
Para a definição da abordagem a ser utilizada, 
três aspectos foram analisados: o contexto de negócio no qual este trabalho está inserido,
a equipe e a interação entre a equipe e o cliente.

No contexto de negócio foram analisadas algumas características como: tamanho da empresa,
estabilidade da empresa e demanda por maior formalização.

A empresa Matriz - Engenharia de Energia por se tratar de uma empresa júnior consiste
em uma empresa pequena, que possui poucas pessoas envolvidas e não demanda grande formalização.

Segundo Alves e Alves (\citeyear{alves}), ``as metodologias tradicionais têm uma orientação
para que as atividades e artefatos sejam formalmente documentados e controlados. Por outro lado, as metodologias ágeis têm seu foco
na iteratividade dos interessados no projeto.'' Dessa forma, como a empresa não demanda muitas formalizações, a abordagem adaptativa, ou ágil,
se adequa a esse perfil.

No aspecto de equipe, características como tamanho da equipe, experiências anteriores e interação semanal 
foram levadas em consideração. A equipe, formada por quatro integrantes, possui pouca experiência 
em Engenharia de Requisitos tanto na abordagem tradicional, quanto na adaptativa. 
As interações entre os membros da equipe foram definidas em quatro vezes na semana.
Alves e Alves (\citeyear{alves}) relata que os valores ágeis estão alinhados com dinâmicas de pequenas e médias organizações.

Para o aspecto de interação com o cliente foi levada em consideração a quantidade de
reuniões predefinidas, que consiste em uma reunião semanal. Desse modo, também analisando 
a proximidade com o cliente.

Os processos ágeis são caracterizados por Pressman (\citeyear{pressman}) como um
processo com capacidade de adaptação e que seja incremental, tendo em vista as
mudanças e o entendimento dos requisitos, e a participação direta do cliente através
do seu constante \textit{feedback}. A interação semanal estabelecida com o cliente permite esse constante
\textit{feedback}.

Leffingwell (\citeyear{safe}) afirma que a metodologia ágil se concentra em dois aspectos principais: 
melhor compreensão das necessidades e construção de software de qualidade. 

Fazer essa afirmação não significa dizer que as metodologias tradicionais não possuem essas características. O processo unificado, 
tem algumas características semelhantes com a metodologia adaptativa, como: ser baseado em interações e estabelecer interação com o cliente
,mas mesmo assim, nesse tipo de abordagem a concentração acaba sendo no processo. (~\cite{safe} ; ~\cite{vasco})

A escolha do \textit{framework} SAFe foi realizada devido ao seu processo de Engenharia de Requisitos (ER), pois o SAFe engloba as atividades
fundamentais da ER. E de acordo com Leffingwell (\citeyear{safe}), foi um modelo criado para suportar a necessidade de \textit{feedback}
para o negócio. Todavia o SAFe será utilizado apenas como um \textit{framework} estrutural que definirá os artefatos e papéis e o Scrum guiará
a execução do trabalho através dos marcos de tempo, de validação e os eventos, como: Reunião de Planejamento da \textit{Sprint} e 
Reunião de Retrospectiva da \textit{Sprint}, como proposto por Shwaber e Sutherland (\citeyear{scrum}).

O processo unificado embora se adeque a alguns aspectos analisados, como analisado por Vasco et al (\citeyear{vasco}),
essa abordagem é melhor  aplicável em grandes projetos e tem tarefas e papéis muito bem definidos.
Já o SAFe e o Scrum permitem uma maior dinâmica na equipe.


