\chapter[Justificativa]{Justificativa da Abordagem}

Para a definição da abordagem a ser utilizada, três aspectos foram analisados:

\begin{itemize} 
 \item O contexto de negócio no qual este trabalho está inserido;
 \item A equipe;
 \item Interação entre a equipe e o cliente.
\end{itemize}


No contexto de negócio foram analisadas algumas características como: tamanho da empresa,
estabilidade da empresa e demanda por maior formalização.

A empresa Matriz - Engenharia de Energia por se tratar de uma empresa júnior consiste
em uma empresa ainda instável. Possui poucas pessoas envolvidas, dessa forma, caracterizando uma empresa
pequena e que não demanda grande formalização.

No aspecto de equipe, características como experiências anteriores e interação semanal 
foram levadas em consideração. A equipe possui pouca experiência 
em Engenharia de Requisitos tanto na abordagem tradicional, quanto na adaptativa. 
As interações entre os membros da equipe foram definidas em quatro vezes na semana.

Para o aspecto de interação com o cliente foi levada em consideração a quantidade de
reuniões predefinidas, que consiste em uma reunião semanal. Desse modo, também analisando 
a proximidade com o cliente.

Considerando os aspectos supracitados, a abordagem definida foi a adaptativa com o uso da 
metodologia Scaled Agile Framework (SAFe) por possuir características adequadas ao perfil
da empresa. A interação com o cliente semanal permite também uma maior comunicação o que
facilita a aplicação da abordagem adaptativa. 

Como contribuição extra, a utilização da metodologia SAFe, permite aos membros vivenciar
o uso de uma metodologia que a equipe não possui experiência.