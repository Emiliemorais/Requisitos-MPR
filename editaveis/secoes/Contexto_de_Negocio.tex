\chapter[Contexto da Empresa]{Contexto da Empresa}
O contexto proposto, se refere a empresa júnior Matriz - Engenharia de Energia, que é 
formada por alunos da Engenharia de Energia da Universidade de Brasilia, Campus Gama. 
O objetivo principal da Matriz é a iniciação dos alunos do curso no mercado de trabalho 
fornecendo aos clientes soluções inteligentes referentes a energia e sustentabilidade.

A empresa visa atender a demanda de residências, condomínios, empresas de médio 
e pequeno porte. Inicialmente, imaginou-se que a Matriz era a fornecedora dos projetos a 
serem trabalhados que possuiam relação com viabilidade energética e soluções sustentáveis,
mas logo descobriu-se que a mesma só poderia fornecer pré-projetos, ela não fornecia 
soluções. Com o passar do tempo, viu-se também que os alunos não possuem o título do CREA 
(Conselho Regional de Engenharia e Agronomia) o que inviabiliza também que os mesmos forneçam 
pré-projetos. No momento, a Matriz apenas realiza o contato com clientes e indica empresas 
para a solução do problema proposto.

\section{O Problema}
Como já dito, a Matriz vêm com a proposta de ser uma Empresa Júnior que proporcione para 
os alunos de graduação uma experiência empresarial. Por ser nova, a empresa não possui 
um processo bem definido que resulte na aquisição de novos clientes. Inicialmente o que 
foi definido, consistia em investir no uso de redes sociais para difundir suas ideias, 
no estudo de mercado no que tange empresas que mais gastam energia e enviar via telefone
propostas dos serviços que a empresa disponibiliza. Feito isso, optou-se então pelo uso de
cartazes nos corredores da Faculdade do Gama (FGA), porém nenhuma das estratégias seguidas
surtiu efeito no que diz respeito a adesão de sócios e clientes.

\section{Possível Solução}
A partir deste quadro, o grupo da disciplina de Requisitos de Software, junto com um 
grupo da disciplina de Modelagem de Processos, foi encubido de criar estratégias 
criativas para obter uma maior adesão a Matriz. O problema foi desenvolido
usando conceitos referentes a Engenharia de Requisitos, juntos a metodologias de 
desenvolvimento adaptativas, desenhando assim um processo que consistiu desde o nível 
de análise do que de fato se necessita para reverter esse quadro, até o ponto de 
apresentar uma solução em software para o problema proposto.

Percebeu-se então que a Matriz não possuia uma forma facilitada de acompanhar o andamento 
do processo. Problema esse que que poderia estar relacionado com a organização, gestão e 
orientação da empresa. Com isso, percebia-se que qualquer consulta documental era feita 
de forma demorada, fato esse que possui relação com uma mão de obra não muito qualificada, 
uma vez que se trata de profissionais ainda graduandos. 

Com esse mapeamento feito, espera-se que com um processo melhor desenhado e mais fácil de ser
acompanhado, a Matriz possua uma melhor abordagem e forma de lidar com clientes, e que de fato
a mesma possa conquistar novas pessoas para melhor desenvolver os alunos, e fornecer seus serviços.
