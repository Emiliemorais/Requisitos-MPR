\section{Papéis}

\begin{itemize}
  \item \textit{Product Manager} (PM)
 \subitem O \textit{Product Manager} consiste na equipe de Modelagem de Processos e foi escolhido
 para assumir também as responsabilidades do \textit{Product Owner} (PO). Essa escolha foi realizada
 levando em consideração que o PM está em um nível acima do PO. Com base no SAFe \cite{safe}, para
 esse processo suas responsabilidades são: \\
  \begin{enumerate}
    \item Definir o Tema de investimento;
    \item Levantar os Épicos;
    \item Levantar as \textit{Features};
    \item Definir o visão; 
    \item Criar o \textit{Roadmap}.
    \item Gerenciar a \textit{Release}; 
    \item Escrever as histórias;
    \item Validar os incrementos de software;
  \end{enumerate}
      
  \item \textit{Scrum Master}
 \subitem O  \textit{Scrum Master} é um membro da equipe de Requisitos. Esse papel foi escolhido
 para auxiliar o PM em suas atividades considerando que a equipe alocada para o papel não possui
 conhecimento sobre Requisitos. Assim, com base no Guia Scrum \cite{scrum}, para
 esse processo esse papel é responsável por: \\
    \begin{enumerate}
     \item Verificar se o processo está sendo realizado conforme planejado;
     \item Auxiliar o PM nas suas atividades; 
     \item Servir de elo de ligação entre o PM e o Time. 
    \end{enumerate}

  \item Time
 \subitem O  Time consiste na equipe de Requisitos e com base nas definições de \citeonline{safe}
 para esse processo, o time é responsável por: \\
    \begin{enumerate}
      \item Acompanhar as atividades de engenharia de Requisitos;
      \item Escrever os testes de aceitação das histórias;
      \item Desenvolver as histórias. 
    \end{enumerate}
      
\end{itemize}
